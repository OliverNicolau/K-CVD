\documentclass[11pt,a4paper,english]{article}
\usepackage[T1]{fontenc}
\usepackage[utf8]{inputenc}
\usepackage{babel}
\usepackage{geometry}
    \geometry{
    	left=2cm,
	right=2cm,
	includeheadfoot, top=1cm, bottom=1cm,
    headsep=2cm,
    footskip=2.5cm
    }
\usepackage{fancyhdr}
\usepackage{lipsum}
\usepackage{import}

% Extensions file
\import{"/Users/aron/Google Drive/TEMPLATES/"}{falcon.tex} % path filled by fasttex in .zprofile

% Setup for page layout (fancyhdr)
\fancyhf{}
\lhead{Aron Fiechter}
\chead{Analitic geometry}
\rhead{\today}
\cfoot{\thepage}
\pagestyle{fancy}

\begin{document}

    {\centering\huge\textbf{Parabola construction and intersection}\par}

    \vspace{1cm}

    \section*{Parabola equation}
    The equation of the parabola \(\phi\) has the form:
    \[
    	\phi =: \{\;(x,y) \St rx^{2} + sy^{2} + txy + ux + vy + w = 0\;\}
    \]

    \section*{Line equation and point coordinates}
    The equation of a line \(l\) is given as:
    \[
    	l =: \{\;(x,y) \St ax + by + c = 0\;\}
    \]
    the point coordinates for the focus are:
    \[
    	f(f_{x},f_{y})
    \]

    \section*{Definition of a parabola}
    A parabola is defined as the set of all points \((x,y)\) that are equidistant from a line (the directrix \(l\)) and a point (the focus \(f\)):
    \[
    	\phi := \{\;(x,y) \St d\left((x,y),l\right) = d\left((x,y),f\right)\;\}
    \]
    therefore we can find the equation of a parabola in function of its directrix and focus.
    \begin{align*}
    d\left((x,y),l\right) &= d\left((x,y),f\right) \\
	\left[d\left((x,y),l\right)\right]^2 &= \left[d\left((x,y),f\right)\right]^2 \\
	\left(\frac{\abs{ax + by + c}}{\sqrt{a^{2} + b^{2}}}\right)^{2} &= \left(\sqrt{\abs{x - f_{x}}^{2} + \abs{y - f_{y}}^{2}}\right)^{2} \\
	a^{2}x^{2} + b^{2}y^{2} + 2abxy + 2acx + 2bcy + c^{2} &= \left(x^{2} - 2f_{x}x + f_{x}^{2} + y^{2} - 2f_{y}y + f_{y}^{2}\right) \cdot \left(a^{2} + b^{2}\right)
	\end{align*}
	\[
		2abxy + 2acx + 2bcy + c^{2} = b^{2}x^{2} + a^{2}y^{2} + (-2a^{2}f_{x}-2b^{2}f_{x})x + (-2a^{2}f_{y}-2b^{2}f_{y})y + a^{2}f_{x}^{2} + a^{2}f_{y}^{2} + b^{2}f_{x}^{2} + b^{2}f_{y}^{2}
	\]
	\[
		-b^{2}x^{2} + -a^{2}y^{2} + (2ab)xy + (2ac-2a^{2}f_{x} + 2b^{2}f_{x})x + (2bc-2a^{2}f_{y} + 2b^{2}f_{y})y + c^{2} - a^{2}f_{x}^{2} - a^{2}f_{y}^{2} - b^{2}f_{x}^{2} - b^{2}f_{y}^{2} = 0
	\]
	\ppar\bigskip

	We get an equation for the parabola \(\;\;
		\phi : rx^{2} + sy^{2} + txy + ux + vy + w = 0
	\;\;\)
	with coefficients:

	\begin{itemize}[label=\(\triangleright\)]\setlength{\itemsep}{-2pt}
    \item \(r = -b^{2}\)
    \item \(s = -a^{2}\)
    \item \(t = 2ab\)
    \item \(u = 2ac-2a^{2}f_{x} + 2b^{2}f_{x}\)
    \item \(v = 2bc-2a^{2}f_{y} + 2b^{2}f_{y}\)
    \item \(w = c^{2} - a^{2}f_{x}^{2} - a^{2}f_{y}^{2} - b^{2}f_{x}^{2} - b^{2}f_{y}^{2}\)
    \end{itemize}
    
    \newpage
    
    \section*{Parabola--line intersection}
    The intersections of a line and a parabola can be found by solving the equation of the line for \(y\), then inserting it into the equation of the parabola to get a quadratic equation in \(x\). If the coefficient \(b\) of the line is equal to 0, instead we can substitute \(x\) in the parabola equation and get a quadratic equation in \(y\).\par
    The resulting \(x\)(s) (or \(y\)(s)) of the equation can be input in the line equation to get the corresponding value(s) of \(y\) (or \(x\)), each indicating a point of intersection.\par
    There can be 0, 1, or 2 intersections.
    \ppar
    If \(b \neq 0\):
    
    \[
    \phi \cap l :=
    \]
    \[
    \begin{cases}
    	0 &=\;\; rx^{2} + sy^{2} + txy + ux + vy + w \\
		0 &=\;\; ax + by + c
    \end{cases} \Iff
    \begin{cases}
    	0 &=\;\; rx^{2} + sy^{2} + txy + ux + vy + w \\
		y &=\;\; \displaystyle\frac{-c}{b} + \displaystyle\frac{-ax}{b}
    \end{cases} \Iff
    \]
    \begin{align*}
    rx^{2} + s\left(\frac{-c}{b} + \frac{-ax}{b}\right)^{2} + tx\left(\frac{-c}{b} + \frac{-ax}{b}\right) + ux + v\left(\frac{-c}{b} + \frac{-ax}{b}\right) + w &= 0
    \\
	rx^{2} + s\left(\frac{(-c) + (-ax)}{b}\right)^{2} + tx\left(\frac{(-c) + (-ax)}{b}\right) + ux + v\left(\frac{(-c) + (-ax)}{b}\right) + w &= 0
	\\
	rx^{2} + s\frac{\big((-c) + (-ax)\big)^{2}}{b^{2}} + \frac{-ct}{b}x + \frac{-at}{b}x^{2} + ux + \frac{-cv}{b} + \frac{-av}{b}x + w &= 0
	\\
	rx^{2} + s\frac{c^{2} + 2acx + a^{2}x^{2}}{b^{2}} + \frac{-ct}{b}x + \frac{-at}{b}x^{2} + ux + \frac{-cv}{b} + \frac{-av}{b}x + w &= 0
	\\
	rx^{2} + \frac{sc^{2}}{b^{2}} + \frac{2acs}{b^{2}}x + \frac{sa^{2}}{b^{2}}x^{2} + \frac{-ct}{b}x + \frac{-at}{b}x^{2} + ux + \frac{-cv}{b} + \frac{-av}{b}x + w &= 0
	\\
	rx^{2} + \frac{sa^{2}}{b^{2}}x^{2} + \frac{-at}{b}x^{2} + \frac{2acs}{b^{2}}x + \frac{-ct}{b}x + \frac{-av}{b}x + ux + \frac{sc^{2}}{b^{2}} + \frac{-cv}{b} + w &= 0
	\\
	\left(r + \frac{sa^{2}}{b^{2}} + \frac{-at}{b}\right)x^{2} + \left(\frac{2acs}{b^{2}} + \frac{-ct}{b} + \frac{-av}{b} + u\right)x + \left(\frac{sc^{2}}{b^{2}} + \frac{-cv}{b} + w\right) &= 0
	\\
    \end{align*}
    
    \ppar
    Instead if \(b = 0\):
    
    \[
    \phi \cap l :=
    \]
    \[
    \begin{cases}
    	0 &=\;\; rx^{2} + sy^{2} + txy + ux + vy + w \\
		0 &=\;\; ax + c
    \end{cases} \Iff
    \begin{cases}
    	0 &=\;\; rx^{2} + sy^{2} + txy + ux + vy + w \\
		x &=\;\; \displaystyle\frac{-c}{a}
    \end{cases} \Iff
    \]
    \begin{align*}
    r\left(\frac{-c}{a}\right)^{2} + sy^{2} + t\left(\frac{-c}{a}\right)y + u\left(\frac{-c}{a}\right) + vy + w &= 0
    \\
    \frac{rc^{2}}{a^{2}} + sy^{2} + \frac{-ct}{a}y + \frac{-cu}{a} + vy + w &= 0
    \\
    \left(s\right)y^{2} + \left(v + \frac{-ct}{a}\right)y + \left(\frac{rc^{2}}{a^{2}} + \frac{-cu}{a} + w\right) &= 0
    \\
    \end{align*}
\end{document}














