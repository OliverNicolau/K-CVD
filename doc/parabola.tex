\documentclass[11pt,a4paper,english]{article}
\usepackage[T1]{fontenc}
\usepackage[utf8]{inputenc}
\usepackage{babel}
\usepackage{geometry}
    \geometry{
    	left=2cm,
	right=2cm,
	includeheadfoot, top=1cm, bottom=1cm,
    headsep=2cm,
    footskip=2.5cm
    }
\usepackage{fancyhdr}
\usepackage{lipsum}
\usepackage{import}

% Extensions file
\import{"/Users/aron/Google Drive/TEMPLATES/"}{falcon.tex} % path filled by fasttex in .zprofile

% Setup for page layout (fancyhdr)
\fancyhf{}
\lhead{Aron Fiechter}
\chead{Analitic geometry}
\rhead{\today}
\cfoot{\thepage}
\pagestyle{fancy}

\begin{document}

    {\centering\huge\textbf{Parabola equation from directrix and focus}\par}

    \vspace{1cm}

    \section*{Parabola equation}
    The equation of the parabola \(\phi\) has the form:
    \[
    	\phi =: \{\;(x,y) \St rx^{2} + sy^{2} + txy + ux + vy + w = 0\;\}
    \]

    \section*{Line equation and point coordinates}
    The equation of a line \(l\) is given as:
    \[
    	l =: \{\;(x,y) \St ax + by + c = 0\;\}
    \]
    the point coordinates for the focus are:
    \[
    	f(f_{x},f_{y})
    \]

    \section*{Definition of a parabola}
    A parabola is defined as the set of all points \((x,y)\) that are equidistant from a line (the directrix \(l\)) and a point (the focus \(f\)):
    \[
    	\phi := \{\;(x,y) \St d\left((x,y),l\right) = d\left((x,y),f\right)\;\}
    \]
    therefore we can find the equation of a parabola in function of its directrix and focus.
    \begin{align*}
    d\left((x,y),l\right) &= d\left((x,y),f\right) \\
	\left[d\left((x,y),l\right)\right]^2 &= \left[d\left((x,y),f\right)\right]^2 \\
	\left(\frac{\abs{ax + by + c}}{\sqrt{a^{2} + b^{2}}}\right)^{2} &= \left(\sqrt{\abs{x - f_{x}}^{2} + \abs{y - f_{y}}^{2}}\right)^{2} \\
	a^{2}x^{2} + b^{2}y^{2} + 2abxy + 2acx + 2bcy + c^{2} &= \left(x^{2} - 2f_{x}x + f_{x}^{2} + y^{2} - 2f_{y}y + f_{y}^{2}\right) \cdot \left(a^{2} + b^{2}\right)
	\end{align*}
	\[
		2abxy + 2acx + 2bcy + c^{2} = b^{2}x^{2} + a^{2}y^{2} + (-2a^{2}f_{x}-2b^{2}f_{x})x + (-2a^{2}f_{y}-2b^{2}f_{y})y + a^{2}f_{x}^{2} + a^{2}f_{y}^{2} + b^{2}f_{x}^{2} + b^{2}f_{y}^{2}
	\]
	\[
		0 = b^{2}x^{2} + a^{2}y^{2} + (-2ab)xy + (-2ac-2a^{2}f_{x}-2b^{2}f_{x})x + (-2bc-2a^{2}f_{y}-2b^{2}f_{y})y + a^{2}f_{x}^{2} + a^{2}f_{y}^{2} + b^{2}f_{x}^{2} + b^{2}f_{y}^{2} - c^{2}
	\]
	\ppar\bigskip

	We get an equation for the parabola \(\;\;
		\phi : rx^{2} + sy^{2} + txy + ux + vy + w = 0
	\;\;\)
	with coefficients:

	\begin{itemize}[label=\(\triangleright\)]\setlength{\itemsep}{-2pt}
    \item \(r = b^{2}\)
    \item \(s = a^{2}\)
    \item \(t = -2ab\)
    \item \(u = -2ac-2a^{2}f_{x}-2b^{2}f_{x}\)
    \item \(v = -2bc-2a^{2}f_{y}-2b^{2}f_{y}\)
    \item \(w = a^{2}f_{x}^{2} + a^{2}f_{y}^{2} + b^{2}f_{x}^{2} + b^{2}f_{y}^{2} - c^{2}\)
    \end{itemize}
    
    \newpage
    
    \section*{Parabola-line intersection}
    The intersection of a line and a parabola can be found 
    


\end{document}
